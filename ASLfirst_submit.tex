\documentclass[10pt,a4paper]{article}
\usepackage{amssymb}
\usepackage{ctex}
\usepackage{amsmath,ntheorem,lmodern,bm}
\usepackage{amsfonts}
\usepackage{geometry}
\usepackage{longtable}
\usepackage{xcolor}
\usepackage{framed}
\usepackage{tcolorbox}
\usepackage{booktabs}
\usepackage{multicol}
\usepackage{multirow}
\usepackage{mathrsfs}
\usepackage{color,framed}
%\usepackage{hyperref}
%\setCJKmainfont[BoldFont={FZHei-B01}]{FZShuSong-Z01}
%\setCJKfamilyfont{song}{FZShuSong-Z01}
%\setCJKfamilyfont{hei}{FZShuSong-Z01}
%\setCJKfamilyfont{heiti}{FZShuSong-Z01}
%\setCJKfamilyfont{heilight}{FZShuSong-Z01}
%\setCJKfamilyfont{title}{FZShuSong-Z01}
%\setCJKfamilyfont{songbold}{FZShuSong-Z01}
%\hypersetup{pdfpagemode=FullScreen}
\colorlet{shadecolor}{gray!10}
\geometry{a4paper,left=1.5cm,right=1.5cm,top=1.5cm,bottom=1.5cm}
\usepackage{ifxetex,ifluatex}
\ifxetex
  \usepackage{fontspec,xltxtra,xunicode}
  \defaultfontfeatures{Mapping=tex-text,Scale=MatchLowercase}
  \newcommand{\euro}{€}
\else
  \ifluatex
    \usepackage{fontspec}
    \defaultfontfeatures{Mapping=tex-text,Scale=MatchLowercase}
    \newcommand{\euro}{€}
  \else
    \usepackage[utf8]{inputenc}
    \usepackage{eurosym}
  \fi
\fi
\usepackage{color}
\usepackage{fancyvrb}
%\DefineShortVerb[commandchars=\\\{\}]{\|}
\DefineVerbatimEnvironment{Highlighting}{Verbatim}{commandchars=\\\{\}}
% Add ',fontsize=\small' for more characters per line
\newenvironment{Shaded}{\begin{shaded}
}{\end{shaded}}
\newcommand{\KeywordTok}[1]{\textcolor[rgb]{0.00,0.44,0.13}{\textbf{{#1}}}}
\newcommand{\DataTypeTok}[1]{\textcolor[rgb]{0.56,0.13,0.00}{{#1}}}
\newcommand{\DecValTok}[1]{\textcolor[rgb]{0.25,0.63,0.44}{{#1}}}
\newcommand{\BaseNTok}[1]{\textcolor[rgb]{0.25,0.63,0.44}{{#1}}}
\newcommand{\FloatTok}[1]{\textcolor[rgb]{0.25,0.63,0.44}{{#1}}}
\newcommand{\CharTok}[1]{\textcolor[rgb]{0.25,0.44,0.63}{{#1}}}
\newcommand{\StringTok}[1]{\textcolor[rgb]{0.25,0.44,0.63}{{#1}}}
\newcommand{\CommentTok}[1]{\textcolor[rgb]{0.38,0.63,0.69}{\textit{{#1}}}}
\newcommand{\OtherTok}[1]{\textcolor[rgb]{0.00,0.44,0.13}{{#1}}}
\newcommand{\AlertTok}[1]{\textcolor[rgb]{1.00,0.00,0.00}{\textbf{{#1}}}}
\newcommand{\FunctionTok}[1]{\textcolor[rgb]{0.02,0.16,0.49}{{#1}}}
\newcommand{\RegionMarkerTok}[1]{{#1}}
\newcommand{\ErrorTok}[1]{\textcolor[rgb]{1.00,0.00,0.00}{\textbf{{#1}}}}
\newcommand{\NormalTok}[1]{{#1}}
\ifxetex
  \usepackage[setpagesize=false, % page size defined by xetex
              unicode=false, % unicode breaks when used with xetex
              xetex,
              colorlinks=true,
              linkcolor=blue]{hyperref}
\else
  \usepackage[unicode=true,
              colorlinks=true,
              linkcolor=blue]{hyperref}
\fi
\hypersetup{breaklinks=true, pdfborder={0 0 0}}
\setlength{\parindent}{0pt}
\setlength{\parskip}{6pt plus 2pt minus 1pt}
\setlength{\emergencystretch}{3em}  % prevent overfull lines
\setcounter{secnumdepth}{0}

%\EndDefineVerbatimEnvironment{Highlighting}
\newcommand{\code}[1]{\ \tcbox[on line,
arc=3pt,outer arc=3pt,colback=pink!30,
boxsep=0pt,left=3pt,right=3pt,top=3pt,bottom=3pt,
boxrule=0pt]{\ttfamily\color{purple}{#1}}\ }

%\title{概率论第一次作业}
%\date{\today}

\begin{document}


%\section{\LaTeX 直播之七  - 数学公式入门基础配套资料}\label{latex-ux516cux5f0f}

%\section{\centerline {概率论第一次作业}}
%\section{\centerline{概率论第一次作业}}
%\section{概率论第一次作业}
{\centering\section*{Homework I: DATA620013}}
{\centering\section*{Advanced Statistical Learning}}
%\subsection{版本号: V1.6.6}\label{ux7248ux672cux53f7-v1.6.6}
%
%\textbf{编者} \\
%\subsection
\begin{flushright}
	\Large {\text{黄之豪}} \\
	\Large {\text{20110980005}} \\
	\Large {\text{Oct 25th, 2020}} \\
\end{flushright}


%\textbf{编者} \\
%中山大学\ 易鹏\ \\
%华南理工大学\ 关舒文
\subsection{1 Problem 1}
Assume that X is not random, $y=X\beta + \varepsilon$\\
$$ E(d^Ty)=E(d^T(X\beta+\epsilon))=d^TX\beta $$ 
As is given unbiased, $Ed^Ty=Ec^Ty=c^T\beta$, so we have $d^TX=c^T$

\begin{align}
Var(\hat{\beta})
& = E(\hat{\beta}-\beta)(\hat{\beta}-\beta)^T \notag \\
& = E((X^TX)^{-1}X^T(X\beta+\varepsilon)-\beta)((X^TX)^{-1}X^T(X\beta+\varepsilon)-\beta)^T \notag \\
& = E((X^TX)^{-1}X^T\varepsilon))((X^TX)^{-1}X^T\epsilon))^T \notag \\
& = E((X^TX)^{-1}X^T\varepsilon\varepsilon^T X (X^TX)^{-1}) \notag \\
& = (X^TX)^{-1} \notag 
\end{align}
\\
so that, \ $Var(c^T\hat{\beta}) = c^T(X^TX)^{-1}c = d^TX(X^TX)^{-1}X^Td$ \\
As the same, \  $Var(d^Ty) = Var(d^T(X\beta+\varepsilon))=E(d^T(X\beta+\varepsilon)-d^TX\beta)(d^T(X\beta+\varepsilon)-d^TX\beta)^T=d^Td$ \\
We want to show that, $\forall d, \  s.t. Ed^Ty=c^T\beta$ \\
$$ Var(c^T\hat{\beta}) \leq Var(d^Ty) \iff d^Td - d^TX(X^TX)^{-1}X^Td \geq 0 \iff I-X(X^TX)^{-1}X^T \succeq 0 $$
\\
To prove that $I-X(X^TX)^{-1}X^T \succeq 0$, we set $I-X(X^TX)^{-1}X^T = M$
$$ M^2= (I-X(X^TX)^{-1}X^T)(I-X(X^TX)^{-1}X^T)=M $$ 
so that, $I-X(X^TX)^{-1}X^T=M=M^2=M^TM$, $I-X(X^TX)^{-1}X^T$ is a positive semidefinite matrix. The proof is completed.



\subsection{2 Problem 2}
$$\theta=\{\mu_{k},\Sigma \} ,\quad   k=1, \cdots, K $$
$$p_{k}(x_{i};\theta)=Pr(G=k|X=x_{i};\theta)=\frac{f(x|G=k)Pr(G=k)}{Pr(X=x)}\propto f(x|G=k)Pr(G=k) $$
the likelihood of the data is
$$ \mathcal{L}(\theta;x)=\prod_{i=1}^N p_{k}(x_{i};\theta)=\prod_{i=1}^N \frac{f(x|G=k)Pr(G=k)}{Pr(X=x)} $$
the log likelihood of the data is
$$ log(\mathcal{L}(\theta;x))=\sum_{i=1}^{N}log(p_{k_{i}}(x_i;\theta))\propto \sum_{i=1}^{N}log(f_{k_{i}}(x_i)\pi_{k_i})=\sum_{i=1}^{N}log(f_{k_{i}}(x_i))+log\pi_{k_i} $$
where  $x_i$ means the $i_{th}$ sample, $k_i$ means the class of the $i_{th}$ sample and
$$ f_{k}(x)=\frac{1}{(2\pi)^\frac{P}{2}{|\Sigma|}^\frac{1}{2}}e^{-\frac{1}{2}(x-\mu_{k})^T\Sigma^{-1}(x-\mu_{k})} $$
(1) $\hat{\mu_{k}}$\\
Suppose that the $k_{th}$ class has $n_k$ samples

\[
	\frac{\partial log(\mathcal{L}(\theta;x))}{\partial \mu_{k}} = \frac{\partial \sum_{i=1}^{N}log(f_{k_{i}}(x_i))}{\partial \mu_{k}}=\frac{\partial \sum_{j=1}^{n_k}log(f_{k}(x_j))}{\partial \mu_{k}}=\sum_{j=1}^{n_k}\frac{\frac{\partial f_{k}(x_j)}{\partial \mu_{k}}}{f_{k}(x_j)} 
	\tag{1}
\]

$$ 
	\frac{\partial f_{k}(x_j)}{\partial \mu_{k}} = f_k(x_j) (-\frac{1}{2}(\Sigma^{-1}+(\Sigma^{-1})^T)(\mu_{k}-x_j)) = f_k(x_j) (-\Sigma^{-1})(\mu_{k}-x_j)
$$
so that (1) equals

\[ 
\sum_{j=1}^{n_k}(-\Sigma^{-1})(\mu_{k}-x_j)
\tag{2}
\]

let (2) equals to 0, we have
\[ 
\hat{\mu_{k}}=\frac{1}{n_{k}}\sum_{j=1}^{n_k}x_j
\]


(2) $\hat{\Sigma}$

From the matrix derivative formula, we know $\mathrm{d}f=tr(\frac{\partial f^T}{\partial x} \mathrm{d} x) $ \  (*), and $d|X|=|X| \operatorname{tr}\left(X^{-1} d X\right)$ \  (**)

\[ \mathop{\arg\max}_{\Sigma}log(\mathcal{L}(\theta;x)) = \mathop{\arg\max}_{\Sigma} \sum_{i=1}^{N}-\frac{1}{2}log|\Sigma|-\frac{1}{2}(x_i-\mu_{k_i})^T\Sigma^{-1}(x_i-\mu_{k_i})= \mathop{\arg\min}_{\Sigma} \sum_{i=1}^{N}log|\Sigma|+(x_i-\mu_{k_i})^T\Sigma^{-1}(x_i-\mu_{k_i}) \tag{3}\]

the derivative of first item of (3) is
\[ 
\sum_{i=1}^{N}\mathrm{d}log|\Sigma|=\sum_{i=1}^{N}|\Sigma|^{-1}\mathrm{d}|\Sigma|\overset{\text{(**)}}{=}\sum_{i=1}^{N}tr(\Sigma^{-1}\mathrm{d}\Sigma)=tr(N*\Sigma^{-1}\mathrm{d}\Sigma)
\tag{4}
 \]
As the assumption above, all K classes and $n_k$ samples about the $k_{th}$ class, the second item of (3) also equals
\[ 
\sum_{k=1}^{K}\sum_{j=1}^{n_k}(x_j-\mu_{k})^T\Sigma^{-1}(x_j-\mu_{k})
\tag{5}
\]

 
the derivative of second item of (3) is


\begin{align}
	\sum_{k=1}^{K}\sum_{j=1}^{n_k}(x_j-\mu_{k})^T\mathrm{d}\Sigma^{-1}(x_j-\mu_{k}) 
	& = -\sum_{k=1}^{K}\sum_{j=1}^{n_k}(x_j-\mu_{k})^T\Sigma^{-1}\mathrm{d}\Sigma\Sigma^{-1}(x_j-\mu_{k}) \notag \\
	& = tr(-\sum_{k=1}^{K}\sum_{j=1}^{n_k}(x_j-\mu_{k})^T\Sigma^{-1}\mathrm{d}\Sigma\Sigma^{-1}(x_j-\mu_{k})) \notag \\
	& = -\sum_{k=1}^{K}\sum_{j=1}^{n_k}tr((x_j-\mu_{k})^T\Sigma^{-1}\mathrm{d}\Sigma\Sigma^{-1}(x_j-\mu_{k}))	\notag \\
	& = -\sum_{k=1}^{K}\sum_{j=1}^{n_k}tr(\Sigma^{-1}(x_j-\mu_{k})(x_j-\mu_{k})^T\Sigma^{-1}\mathrm{d}\Sigma)	\notag \\
	& = -\sum_{k=1}^{K}tr(\Sigma^{-1}\sum_{j=1}^{n_k}(x_j-\mu_{k})(x_j-\mu_{k})^T\Sigma^{-1}\mathrm{d}\Sigma)	\tag{6}
\end{align}

let $S_k=\sum_{j=1}^{n_k}(x_j-\mu_{k})(x_j-\mu_{k})^T$, the (6) equals 
\[ 
-\sum_{k=1}^{K}tr(\Sigma^{-1}S_k\Sigma^{-1}\mathrm{d}\Sigma)
\tag{7}
 \]

from (3), (4) and (7), we have
\[ \mathrm{d}log(\mathcal{L})=tr(N*\Sigma^{-1}\mathrm{d}\Sigma-\Sigma^{-1}S_k\Sigma^{-1}\mathrm{d}\Sigma)=tr((N*\Sigma^{-1}-\Sigma^{-1}\sum_{k=1}^{K}S_k\Sigma^{-1})\mathrm{d}\Sigma) \]

moreover, from (*), we have
\[ \frac{\partial log(\mathcal{L}(\theta;x))}{\partial \Sigma}=N*\Sigma^{-1}-\Sigma^{-1}\sum_{k=1}^{K}S_k\Sigma^{-1}
\tag{8} \]
let (8) equals 0, we have
\[ \hat{\Sigma}=\frac{1}{N}\sum_{k=1}^{K}S_k=\sum_{k=1}^{K}\frac{n_k-1}{N}(\frac{1}{n_k-1}S_k)  \]
\[ S_k=\sum_{j=1}^{n_k}(x_j-\hat{\mu_{k}})(x_j-\hat{\mu_{k}})^T \]

the $ \frac{1}{n_k-1}S_k $ is the covariance matrix of the $k_{th}$ class, so the weight of the pooled covariance estimate is $\frac{n_k-1}{N}$


\subsection{3 Problem 3}
First solve the first principle component $ y_1=\alpha_1^T\mathbf{x} $, as $ var(y_1)=\alpha_1^T \Sigma \alpha_1 $, that is to solve the optimization problem as follow,

\begin{alignat*}{2}
	\max_{\alpha_1} \quad & \alpha_1^T \Sigma \alpha_1 \\
	\mbox{s.t.}\quad
	&\alpha_1^T\alpha_1 = 1 
\end{alignat*}

To solve the optimization problem, a Lagrange function should be defined as follow
\[ \mathcal{L}(\alpha_1, \lambda)= \alpha_1^T \Sigma \alpha_1 - \lambda(\alpha_1^T\alpha_1 - 1 ) \]

\[ \mathcal{L}'_{\alpha_1}= \Sigma \alpha_1 - \lambda \alpha_1=0 \]
so that $\lambda$ is the eigenvalue of $\Sigma$, and $\alpha_1$ is the corresponding eigenvector. The objective function becomes
\[ \alpha_1^T \Sigma \alpha_1=\alpha_1^T \lambda \alpha_1=\lambda \alpha_1^T  \alpha_1=\lambda \]
If we want to maximize the objective function, we should let $\lambda$ to be the max of all eigenvalues of $\Sigma$ , which is $\lambda_1$
\[ var(y_1)=\alpha_1^T \Sigma \alpha_1=\lambda_1 \]

Second, solve the second principle component 

\begin{alignat*}{2}
	\max_{\alpha_2} \quad & \alpha_2^T \Sigma \alpha_2 \\
	\mbox{s.t.}\quad
	&\alpha_2^T\Sigma\alpha_2 = 1 \\
	&\alpha_1^T\Sigma\alpha_2 = 0,  \alpha_2^T\Sigma\alpha_1 = 0
\end{alignat*}

note that $ \alpha_1^T\Sigma\alpha_2 = \alpha_2^T\Sigma\alpha_1 =\alpha_2^T\lambda_1\alpha_1=0 $ \\
so that 
\[ \alpha_2^T\alpha_1=\alpha_1^T\alpha_2=0 \]
we have the Lagrange function as 
\[ \mathcal{L}(\alpha_2, \lambda, \mu)= \alpha_2^T \Sigma \alpha_2 - \lambda(\alpha_2^T\alpha_2 - 1 ) - \mu(\alpha_2^T\alpha_1) \]

\[ \mathcal{L}'_{\alpha_2}= 2\Sigma \alpha_2 - 2\lambda \alpha_2 - \mu\alpha_1=0 \]
let $\alpha_1^T$ left multiply the above formula, we have
\[ 2\alpha_1^T \Sigma \alpha_2 - 2\lambda\alpha_1^T \alpha_2 - \mu\alpha_1^T\alpha_1=0 \]
so that $\mu=0$,and we have
\[ \Sigma\alpha_2-\lambda\alpha_2=0 \]
so that we know $\lambda_2$ is the second largest eigenvalue, and 
\[ var(y_2)=\alpha_2^T \Sigma \alpha_2=\lambda_2 \]
Then use Recursion method, we have
\[ var(y_k)=\alpha_k^T \Sigma \alpha_k=\lambda_k \]
where $ \lambda_k $ is the $k_{th}$ largest eigenvalue.

\subsection{4 Problem 4}
\[ p(\beta|x)=\frac{p(x|\beta)p(\beta)}{p(x)} \propto p(x|\beta)p(\beta) \] 
\[ \mathop{\arg\max}_{\beta}p(\beta|x)=\mathop{\arg\max}_{\beta} p(x|\beta)p(\beta) \] 
let $\operatorname{Pr}(G=1 \mid X=x, \beta) = \pi(x)$, so $\operatorname{Pr}(G=0 \mid X=x, \beta) = 1- \pi(x)$

\begin{align}
	\mathop{\arg\max}_{\beta} p(x|\beta)p(\beta)
	& = \mathop{\arg\max}_{\beta} \prod_{i=1}^{n}p(x_i|\beta)p(\beta) \notag \\
	& = \mathop{\arg\max}_{\beta} \prod_{i=1}^{n}{\pi(x_i)}^{y_i}[1-\pi(x_i)]^{(1-y_i)}p(\beta) \notag \\
	& = \mathop{\arg\max}_{\beta}\sum_{i=1}^{n}y_i log\pi(x_i)+(1-y_i)log(1-\pi(x_i))+logp(\beta)	\notag \\
	& = \mathop{\arg\max}_{\beta}\sum_{i=1}^{n}y_i log\pi(x_i)+(1-y_i)log(1-\pi(x_i))	\notag 
\end{align}

so the loss function can be supposed to
\[ \mathcal{L}(\beta)=-\sum_{i=1}^{n}y_i log\pi(x_i)+(1-y_i)log(1-\pi(x_i))= -\sum_{i=1}^{n}y_i log\frac{\pi(x_i)}{1-\pi(x_i)}+log(1-\pi(x_i))= -\sum_{i=1}^{n}[y_i (\beta^Tx_i)-log(1+e^{\beta^Tx_i})] \]




\end{document} 